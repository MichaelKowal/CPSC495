\documentclass[12pt, oneside]{article}   	% use "amsart" instead of "article" for AMSLaTeX format
\usepackage{geometry}                		% See geometry.pdf to learn the layout options. There are lots.
\geometry{letterpaper}                   		% ... or a4paper or a5paper or ... 
\usepackage{graphicx}				% Use pdf, png, jpg, or eps§ with pdflatex; use eps in DVI mode
								% TeX will automatically convert eps --> pdf in pdflatex		
\usepackage{amssymb}
\usepackage{lmodern}

\title{Combining ShiViz and Distributed System Virtualization Software}
\author{Michael Kowal}


\begin{document}
\begin{titlepage}
	\centering
	\vspace{10cm}
	{\scshape\huge  Combining ShiViz and Distributed System Virtualization Software\par}
	\vspace{1.5cm}
	{\scshape\LARGE CPSC 495 Report\par}
	\vspace{1.5cm}
	{\Large\itshape Michael Kowal\par}
	\vfill
\end{titlepage}

\thispagestyle{empty}
\tableofcontents
\newpage
\clearpage
\pagenumbering{arabic} 
\setlength{\columnsep}{1 cm}
\begin{twocolumn}
\section{Introduction}

The purpose of this project was to combine two distributed systems applications and provide ways for users to easily use the modifications.  All of the modifications are done to UNBC's software, ShiViz was never changed.  The project took approximately one and a half months to complete, excluding the documentation.  

\section{Vector Clocks}

The primary relationship between the two programs is their vector clocks.  Both of them track events based on a vector clock that records which node in a collection of nodes is currently preforming an action.  The clock will increment for that node upon completion of that action

\section{ShiViz}

ShiViz is a software for visualizing data flow through distributed systems.  It was created by students at the University of British Columbia.  The tool helps users understand the flow of a concurrent or distributed system by providing users with an interactive graph that displays where data travels and when.  It allows users to see patterns, called "motifs" in the data flow as well as provides many different ways to search and manipulate the graph to see more specific events.  The software requires specially formatted log files that, when input into the site, get interpreted and displayed correctly as an interactive graph.

\section{Distributed System\\Virtualization}

The primary software that was used in this project is the distributed system virtualization software created by students at the University of Northern British Columbia.  This software gives users the ability to design a distributed system and test that system without having to pay the huge costs of buying all the necessary hardware.  The tool provides many different pre-made topologies that can be modified by the user as well as the option to draw an entire network from scratch.  The user creates nodes and then assigns connections between those nodes.  It is then possible to adjust the behaviour of each individual node as well as apply different settings to the entire network. ...

\section{Benefits of Integration}

By combining these two softwares, users will be able to get even more information on the viability of their distributed system.  ShiViz will be able to visualize exactly what is happening in the virtualized distributed system and users will be able to make improvements if need be.

\section{Changes Made}

There were not many adjustments that needed to be made to the distributed system software in order to get it to create a log file that ShiViz would be able to understand.  After importing the .jar file provided by ShiViz, it was only a matter of finding out where to call the log maker.  The ShiViz package provides a proper vector clock of its own so it was able to keep things organized with only a host name and event. ...

\section{How to Use the Modified Software}

It is very easy to use the software.  A log is created every time a simulation is run.  In order to see the ShiViz visualization, a user simply has to click the "Open ShiViz" button on the "Simulation" page of the DS software, then select the log they wish to use from the "LogFiles" folder that is found in the same location as the DS software .jar.

\section{Challenges Faced}

By far the biggest issue with this project was figuring out how to properly format the received message in order to pass it into the log builder.  Significantly more time was spent on that than anything else.  What eventually worked was to take the original sent message, save it, and modify it slightly as the received message once it reached its destination.

\end{twocolumn}
\end{document}   



